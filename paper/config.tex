% !TEX root = main.tex
% chktex-file 46

% **************************************************
% Files' Character Encoding
% **************************************************
\PassOptionsToPackage{utf8}{inputenc}
\usepackage{inputenc}
\usepackage[ngerman,english]{babel}

% **************************************************
% Information and Commands for Reuse
% **************************************************
\newcommand{\thesisTitle}{Spectral Graph Approximation}
\newcommand{\thesisName}{Clemens Damke}
\newcommand{\thesisMatNr}{7011488}
\newcommand{\thesisSubject}{Seminar paper}
\newcommand{\thesisDate}{\today}
\newcommand{\thesisVersion}{Draft}

\newcommand{\thesisSupervisor}{Vitalik Melnikov}

\newcommand{\thesisUniversity}{Paderborn University}
\newcommand{\thesisUniversityDepartment}{Department of Computer Science}
\newcommand{\thesisUniversityInstitute}{Heinz Nixdorf Institute}
\newcommand{\thesisUniversityGroup}{Intelligent Systems and Machine Learning Group (ISG)}
\newcommand{\thesisUniversityCity}{Paderborn}
\newcommand{\thesisUniversityStreetAddress}{Warburger Straße 100}
\newcommand{\thesisUniversityPostalCode}{33098}



% **************************************************
% Debug LaTeX Information
% **************************************************
%\listfiles


% **************************************************
% Load and Configure Packages
% **************************************************
\usepackage{geometry}
\geometry{
  a4paper,         % or letterpaper
  textwidth=13cm,  % llncs has 12.2cm
  textheight=23cm, % llncs has 19.3cm
  heightrounded,   % integer number of lines
  hratio=1:1,      % horizontally centered
  vratio=1:1,      % vertically centered
}
\renewcommand{\baselinestretch}{1.15}
\usepackage[parfill]{parskip}

% Colors:
\usepackage[usenames, dvipsnames, svgnames, table]{xcolor}

\definecolor{schwarz}{HTML}{000000}
\definecolor{blau}{HTML}{355FB3}
\definecolor{rot}{HTML}{B33535}
\definecolor{gruen}{HTML}{3BB335}
\definecolor{dunkelblau}{HTML}{1E3666}
\definecolor{hellblau}{HTML}{8ea7d7}

\usepackage{mathtools}
\usepackage{bm}
\usepackage{bbm}
\usepackage{units}
\newcommand\numberthis{\addtocounter{equation}{1}\tag{\theequation}}

\usepackage{algpseudocode}

\usepackage{graphicx}
\usepackage{tikz}
\usetikzlibrary{arrows,positioning}
\usetikzlibrary{calc}
\newcommand{\tikzmark}[1]{\tikz[overlay,remember picture] \node (#1) {};} % chktex 1
\usepackage[labelfont=bf]{caption}
\usepackage{subcaption}

\usepackage{pgfplots}
\usepackage{pgfplotstable}
\pgfplotsset{compat=1.14}
\usepgfplotslibrary{dateplot, statistics}
\pgfplotsset{
    cycle list={blau\\rot\\gruen\\schwarz\\},
}

\usepackage{listings}
\lstset{basicstyle=\ttfamily,breaklines=true}

\usepackage{tasks}
\settasks{counter-format=tsk[1].}

\usepackage[bguq]{frege}
\usepackage{stmaryrd}
\usepackage{multicol}
\usepackage{pbox}
\usepackage{longtable}
\usepackage{booktabs}
\usepackage{csvsimple}
\usepackage{siunitx}

\usepackage{hyperref}
\hypersetup{% setup the hyperref-package options
    pdftitle={\thesisTitle},    %   - title (PDF meta)
    pdfsubject={\thesisSubject},%   - subject (PDF meta)
    pdfauthor={\thesisName},    %   - author (PDF meta)
    plainpages=false,           %   -
    colorlinks=false,           %   - colorize links?
    pdfborder={0 0 0},          %   -
    breaklinks=true,            %   - allow line break inside links
    bookmarksnumbered=true,     %
    bookmarksopen=true          %
}

\usepackage[						% use biblatex for bibliography
	backend=bibtex,					% 	- use biber backend (bibtex replacement) or bibtex
	style=numeric,					% 	- use alphabetic (or numeric) bib style
	natbib=true,					% 	- allow natbib commands
	hyperref=true,					% 	- activate hyperref support
	backref=true,					% 	- activate backrefs
	isbn=false,						% 	- don't show isbn tags
	url=false,						% 	- don't show url tags
	doi=false,						% 	- don't show doi tags
	urldate=long,					% 	- display type for dates
	maxnames=3,%
	minnames=1,%
	maxbibnames=5,%
	minbibnames=3,%
	maxcitenames=2,%
	mincitenames=1%
]{biblatex}
\bibliography{bib-refs}

\DeclareCiteCommand{\citenum}
  {}
  {\bibhyperref{\printfield{labelnumber}}}
  {}
  {}

\makeatletter
\newcounter{rulecount}[section]
\newcommand{\ruleno}[1]{\ensuremath{r_{#1}}}
\newcommand{\rulecurrent}{\ruleno{\therulecount}}
\newcommand{\rulemark}[1]{\refstepcounter{rulecount}(\rulecurrent)\ltx@label{#1}} % chktex 36
\newcommand{\ruleref}[1]{\ruleno{\ref{#1}}}
\makeatother
