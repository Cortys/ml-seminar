% !TEX root = ../main.tex
% chktex-file 21
% chktex-file 46
\section{Conclusion}%
\label{sec:conclusion}

We have now discussed the three main topics of this paper:
\begin{enumerate*}
	\item How the structural properties of graphs can be described via spectral graph theory.
	\item How graphs can be coarsened via the REC algorithm.
	\item How to bound the effects of coarsening via RSS and what this implies for spectral clustering.
\end{enumerate*}


Based on the work we presented, there are two main open questions for future research:
\begin{enumerate*}
	\item The described RSS bound assumes a single application of the REC algorithm.
		For multiple REC applications with a total reduction ratio of $r > \frac{1}{2}$, the RSS bound has to be adapted.
	\item Currently the RSS bound has only been applied to the analysis of spectral clustering.
		To make the results more generally applicable the implications for other graph algorithms, graph convolutional neural networks in particular, are considered by \citet{Loukas2018}.
\end{enumerate*}
