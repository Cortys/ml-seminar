% !TEX root = ../main.tex
% chktex-file 21
\section{Introduction}%
\label{sec:intro}

\pagenumbering{arabic}			% arabic page numbering
\setcounter{page}{1}			% set page counter

With the rise of Big Data applications over the recent years, working with large graph structures also became more important.
Algorithms like PageRank or spectral clustering are commonly used to analyze the web graph or social networks.
In order to run such algorithms on large graphs however, optimizations are required.

For this purpose we will specifically look at graph coarsening.
Coarsening reduces the size of a given graph while preserving its overall structure via some notion of graph similarity that will be defined later.
Graph algorithms can then be run on the smaller coarsened graph.
Afterwards the result for the coarsened graph can be iteratively refined to obtain an approximate result for the original graph.
Figure X illustrates this approach.

This paper gives an introduction to graph coarsening and its effects.
This introduction consists of three sections, each of which aims to answer one main question:
\begin{enumerate}
	\item \textbf{Spectral Graph Theory:} How can the structural properties of a graph be formally described?
	\item \textbf{Graph Coarsening:} How can the size of a graph be reduced?
	\item \textbf{Spectral Graph Similarity:} How does graph coarsening affect the structure of a graph and the results of graph algorithms?
\end{enumerate}
